\documentclass{article}

\usepackage{booktabs}
\usepackage{tabularx}
\usepackage{hyperref}
\usepackage{fancyhdr}
\pagestyle{fancy}

\hypersetup{
    colorlinks,
    citecolor=black,
    filecolor=blue,
    linkcolor=red,
    urlcolor=blue
}
\usepackage{color}


\title{SE 3XA3: Development Plan\\Ultimate Tic Tac Toe}

\author{Team 3, Tic Tac Toe
		\\ Kunal Shah | shahk24
		\\ Pareek Ravi | ravip2
}

\date{December 7 2016}

%% Comments

\usepackage{color}

\newif\ifcomments\commentstrue

\ifcomments
\newcommand{\authornote}[3]{\textcolor{#1}{[#3 ---#2]}}
\newcommand{\todo}[1]{\textcolor{red}{[TODO: #1]}}
\else
\newcommand{\authornote}[3]{}
\newcommand{\todo}[1]{}
\fi

\newcommand{\wss}[1]{\authornote{blue}{SS}{#1}}
\newcommand{\ds}[1]{\authornote{red}{DS}{#1}}
\newcommand{\mj}[1]{\authornote{red}{MSN}{#1}}
\newcommand{\mh}[1]{\authornote{red}{MH}{#1}}
\newcommand{\cm}[1]{\authornote{red}{CM}{#1}}


% team members should be added for each team, like the following
% all comments left by the TAs or the instructor should be addressed
% by a corresponding comment from the Team

\newcommand{\tm}[1]{\authornote{magenta}{Team}{#1}}


\begin{document}

\lhead{Team 3 - Development Plan}
\rhead{Ultimate Tic Tac Toe}
\maketitle

\tableofcontents

\newpage

\begin{table}[hp]
\caption{Revision History} \label{TblRevisionHistory}
\begin{tabularx}{\textwidth}{llX}
\toprule
\textbf{Date} & \textbf{Developer(s)} & \textbf{Change}\\
\midrule
September 28 & Pareek Ravi & Initial Setup\\
September 28 & Pareek Ravi and Kunal Shah & Started on Development Plan\\
September 29 & Kunal Shah & Team Meeting Plan, Team Communication Plan and Team Member Roles\\
September 29 & Pareek Ravi & Proof of Concept Demonstration Plan , Gantt Chart\\
September 30 & Kunal Shah & Technology , Git Workflow Plan \\
September 30 & Kunal Shah & Coding Style and updated Technology\\
October 5 & Kunal Shah & Changed Gantt Chart Location\\
December 5 & Pareek Ravi & Made changes based on suggestions from Chris\\

\bottomrule
\end{tabularx}
\end{table}
\newpage

\cm{This text can be under a section names \textit{Abstract}} \\
\section*{Abstract}
Ultimate Tic Tac Toe is a variation on the classic game of Tic Tac Toe. It is 
simply multiple games of Tic Tac Toe running simultaneously to make a classic 
game that often ends in a draw have an exciting ending.

\section{Team Meeting Plan}
Our Team will be meeting 4 times a week; twice during lab hours and twice 
outside of lab. Out of lab meetings will take place in a few locations such as
Thode Library, Health Sciences Library, team member?s homes or via online 
 mediums such as Skype. During each meeting minutes will be taken down. 
 This will recap what was done during the meeting, what we have done since the 
last meeting and what we plan to accomplish before the next meeting.

\section{Team Communication Plan}
All team communication about setting group
meetings and project related communication will occur on the Facebook messenger
group chat. Skype will be used to conduct ''virtual'' face to face meetings
outside of lab. Git issues will be used for setting milestones, delegating tasks
and reporting bugs in the program.

\section{Team Member Roles}
During every meeting one member (alternating) will
be the meeting facilitator. Other roles are distributed
as follows:
\begin{itemize}
  \item Kunal: Developer, LaTex Technology Expert, HTML, CSS Expert 
  \item Pareek: Developer,Git Technology Expert, Gantt Expert, JavaScript Expert 

\end{itemize}

\section{Git Workflow Plan}
Our team will be using a Master only git workflow.
This means all code will be pushed to the master branch. Every milestone commit
will be tagged with its predefined identifier. If code that is currently work in
progress is pushed to the repository, it's commit message must indicate that with the
string "WIP". If there is code that is not working, it must have a git issue
linked to the commit.

\section{Proof of Concept Demonstration Plan}
\cm{Of all the tasks below, which is going to be demonstrated during the Proof of Concept?} \\
The proof of concept demonstration will involve the fundamental game dynamics
fully working. This will include the logic to determine if a move is valid, if a
player has taken control of one inner board and if a player has won the entire
game. It will also have user inputs from the same local machine for up to 5
moves each.

\section{Technology}
Technologies that will be used include:

\begin{itemize}

  \item Git - Project version control
  \item LaTex - Document preparation system
  \item Javascript - Programming language for interactive effects within web browsers
  \item HTML - Standard language for World Wide Web sites.
  \item CSS - HTML Styling Language 
  \item JsDoc - Documentation generation system
  \item Karma - Javascript Unit Testing system 

\end{itemize}

\section{Coding Style}
\cm{Although I do appreciate the link, make sure to include proper punctuation} \\
The project shall be coded using \href{https://google.github.io/styleguide/javascriptguide.xml}
{Google Javascript style guidelines.}

\section{Project Schedule}

\cm{Consider using more appropriate text. With respect to the chart, more details regarding the full implementation would have been better.} \\
A \href{run:../../ProjectSchedule/Gantt Chart.gan}{Gantt Chart} with a detailed breakdown of the milestones of this project.
\todo{Update Gantt Chart}
\section{Project Review}
\todo{Need to do}

\newpage

\end{document}