\documentclass{article}

\usepackage{booktabs}
\usepackage{tabularx}
\usepackage{hyperref}
\usepackage{color}

\title{SE 3XA3: Development Plan\\Ultimate Tic Tac Toe}

\author{Team 3, Tic Tac Toe
		\\ Kunal Shah | shahk24
		\\ Pareek Ravi | ravip2
}

\date{}

%% Comments

\usepackage{color}

\newif\ifcomments\commentstrue

\ifcomments
\newcommand{\authornote}[3]{\textcolor{#1}{[#3 ---#2]}}
\newcommand{\todo}[1]{\textcolor{red}{[TODO: #1]}}
\else
\newcommand{\authornote}[3]{}
\newcommand{\todo}[1]{}
\fi

\newcommand{\wss}[1]{\authornote{blue}{SS}{#1}}
\newcommand{\ds}[1]{\authornote{red}{DS}{#1}}
\newcommand{\mj}[1]{\authornote{red}{MSN}{#1}}
\newcommand{\mh}[1]{\authornote{red}{MH}{#1}}
\newcommand{\cm}[1]{\authornote{red}{CM}{#1}}


% team members should be added for each team, like the following
% all comments left by the TAs or the instructor should be addressed
% by a corresponding comment from the Team

\newcommand{\tm}[1]{\authornote{magenta}{Team}{#1}}


\begin{document}

\maketitle

Ultimate Tic Tac Toe is a variation on the classic game of Tic Tac Toe. It is 
simply multiple games of Tic Tac Toe running simultaneously to make a simple 
game that often ends in a draw have an exciting ending.

\section{Team Meeting Plan}
Our Team will be meeting 4 times a week; twice during lab hours and twice 
outside of lab. Out of lab meetings will take place in a few locations such as
Thode Library, Health Sciences Library, team member?s homes or via online 
 mediums such as Skype. During each meeting minutes will be taken down. 
 This will recap what was done during the meeting, what we have done since the 
last meeting and what we plan to accomplish before the next meeting. \\

\section{Team Communication Plan}
All team communication about setting group meetings and project related 
communication will occur on the FaceBook messenger group chat. Skype will be 
used to conduct ''virtual'' face to face meetings outside of lab. Git issues 
will be used for setting milestones, delegating tasks and reporting bugs in the 
program. \\

\section{Team Member Roles}
During every meeting one member (alternating) will be the meeting facilitator. 
This role consists of fill out the \href{run:3XA3 Meeting Summary Template.docx}{\textcolor{blue}{meeting minutes word document}} 
and saving a PDF copy to the git repository. Other Member roles are distributed as follows:\\
Kunal: Developer, LaTex Technology Expert, HTML, CSS Expert \\
Pareek: Developer,Git Technology Expert, JavaScript Expert \\

\section{Git Workflow Plan}

\section{Proof of Concept Demonstration Plan}\\
The proof of concept demonstration will involve the fundamental game dynamics 
fully working. This will include the logic to determine if a move is valid, if a 
player has taken control of one inner board and if a player has won the entire 
game. It will also have user inputs from the same local machine for up to 5 
moves each. One possible difficulty is the testing with the Karma Unit Testing 
tool as it is something that we have never used before. We hope to implement 
this game using a local server so people can play the game on their own devices. 
This will be an additional feature to add if the time frame permits it because 
of the complexity of it. We do not foresee any difficulties in installing 
external libraries. The aim is to have this application running on all forms of 
devices; mobile, tablet and computer. With modifications to the CSS, this should 
be achievable. \\

\section{Technology}

\section{Coding Style}

\section{Project Schedule}

Pointer to \href{run:Gantt Chart.gan}{\textcolor{blue}{Gantt Chart}}.

\section{Project Review}

\newpage

\begin{table}[hp]
\caption{Revision History} \label{TblRevisionHistory}
\begin{tabularx}{\textwidth}{llX}
\toprule
\textbf{Date} & \textbf{Developer(s)} & \textbf{Change}\\
\midrule
September 28 & Pareek Ravi & Initial Setup\\
September 28 & Pareek Ravi and Kunal Shah & Started on Development Plan\\
September 29 & Kunal Shah & Team Meeting Plan, Team Communication Plan and Team Member Roles\\
September 29 & Pareek Ravi & Proof of Concept Demonstration Plan , Gannt Chart\\
... & ... & ...\\
\bottomrule
\end{tabularx}
\end{table}

\end{document}