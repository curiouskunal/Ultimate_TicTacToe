\documentclass[12pt, titlepage]{article}

\usepackage{booktabs}
\usepackage{tabularx}
\usepackage{hyperref}
\usepackage{fancyhdr}

\pagestyle{fancy}

\hypersetup{
    colorlinks,
    citecolor=black,
    filecolor=blue,
    linkcolor=red,
    urlcolor=blue
}
\usepackage[round]{natbib}
\usepackage{graphicx}

\newcounter{funreq}
\newcommand{\frthefunreq}{FR\thefunreq}
\newcommand{\aref}[1]{FR\ref{#1}}

\newcounter{lfreq}
\newcommand{\lfthelfreq}{LF\thelfreq}
\newcommand{\bref}[1]{LF\ref{#1}}

\newcounter{uhreq}
\newcommand{\uhtheuhreq}{UH\theuhreq}
\newcommand{\cref}[1]{UH\ref{#1}}

\newcounter{perreq}
\newcommand{\pertheperreq}{PR\theperreq}
\newcommand{\dref}[1]{PR\ref{#1}}

\newcounter{oereq}
\newcommand{\oetheoereq}{OE\theoereq}
\newcommand{\eref}[1]{OE\ref{#1}}

\newcounter{msreq}
\newcommand{\msthemsreq}{MS\themsreq}
\newcommand{\fref}[1]{MS\ref{#1}}

\newcounter{secreq}
\newcommand{\sethesecreq}{SE\thesecreq}
\newcommand{\gref}[1]{SE\ref{#1}}

\newcounter{culreq}
\newcommand{\cultheculreq}{CU\theculreq}
\newcommand{\iref}[1]{CU\ref{#1}}

\newcounter{legreq}
\newcommand{\legthelegreq}{LE\thelegreq}
\newcommand{\jref}[1]{LE\ref{#1}}

\newcounter{hsreq}
\newcommand{\hsthehsreq}{HS\thehsreq}
\newcommand{\kref}[1]{HS\ref{#1}}

\title{SE 3XA3: SRS\\Ultimate Tic Tac Toe}

\author{Team 3
		\\ Kunal Shah - shahk24
		\\ Pareek Ravi - ravip2
}

\date{\today}

\input{../Comments}

\begin{document}
\lhead{Team 3 - System Requirement Specification}
\rhead{Ultimate Tic Tac Toe}
\maketitle

\pagenumbering{roman}
\tableofcontents
\listoftables
\listoffigures

\newpage

\begin{table}[hp]
\caption{\bf Revision History}
\begin{tabularx}{\textwidth}{p{3cm}p{2cm}X}
\toprule {\bf Date} & {\bf Version} & {\bf Notes}\\
\midrule
October 3 & 1.0 & initial setup\\
October 5 & 1.1 & updated Project Drivers content\\
October 6 & 1.2 & added Functional Requirements\\
October 6 & 1.3 & added Non-functional Requirements\\
October 7 & 1.4 & updated Functional \& Non-functional Requirements\\
October 7 & 1.5 & Made notes for Project Issues\\
October 11 & 1.5 & finished for Project Issues\\
October 11 & 2.0 & Format document\\
\bottomrule
\end{tabularx}
\end{table}

\newpage

\pagenumbering{arabic}

\section*{Abstract}
This document describes the requirements for Ultimate Tic Tac Toe. The template 
for the Software Requirements Specification (SRS) is a subset of the Volere
template~\citep{RobertsonAndRobertson2012}.

\section{Project Drivers}

\subsection{The Purpose of the Project}
The purpose of this project is to redevelop an existing open source project
while following proper documentation and waterfall design methods.

\subsection{The Stakeholders}

The stakeholders include all parties that would have a vested interest in this
projected. This includes, the client Dr Smith, the developers Pareek and
Kunal ,all the people who would be interested in playing the game and the
teaching assistant Christopher.


\subsubsection{The Client}
The client of Ultimate Tic Tac Toe is Dr. Smith as he is the project manager and 
he requested this project. We are creating this game for him to distribute to 
the customer.

\subsubsection{The Customers}
The target customer for this game would be anyone who has a device connected to
the internet and wants to play a game with a friend. This game is targeted
towards young people, but it is open for anyone to play.

\subsubsection{Other Stakeholders}
Another stakeholder is the teaching assistant, Christopher, as is aiding Dr.
Smith in project management during lab and developers during the development 
process.

\subsection{Mandated Constraints}
There are 3 constraints of this game.	
\begin{enumerate}
	\item The internet
  	\item A device that can connect to the internet
  	\item A friend to play the game with
\end{enumerate}

\subsection{Naming Conventions and Terminology}
The naming convention for winning an inner game will be called controlling a
square. In order to win, a player must get a tic tac toe of controlled squares.
If there is a draw in a square, it will be called a dumped square.

\subsection{Relevant Facts and Assumptions}
We assume that the users have a stable internet connection and are aware of the
rules of regular Tic Tac Toe. It is important to note that the game will work
better in large screens rather than a small display. We assume that the user is
using a device that is either touch enables or has a mouse to click with.

\section{Functional Requirements}

\subsection{The Scope of the Work and the Product}

We investigated the original Ultimate tic tac toe game ~\citep{githubREF}. In
this  game the developer has poor user interface and no game instructions. It
also only supports single device with no AI option. These shortcomings are
the features we would like to improve upon.

\subsubsection{The Context of the Work}
The central task is to create a game of Ultimate Tic Tac Toe that can be played
on the internet with a friend. This will require a server to host the game, each
user to connect to the game, and ensure a connection with the server. In order
for the game to be played, the server will determine which player is to play,
get that player's move record it in the virtual board, and then notify the 
other player of the first player's move. This process of sending the move made 
back and forth will be the role of the server. The server will also indicate to
a player if they have won, lost or if that game has ended in a draw.

\subsubsection{Work Partitioning}
The development of this game is shared evenly between the two team members. The
game logic is developed by both members. The graphical portion is mainly going
to be done by Kunal, and the server portion will be done by Pareek.

\subsubsection{Individual Product Use Cases}
One of a possible use case is when a player might disconnect from their
internet connection, but their internet resumes and they wish to reconnect to
the game. Another case is if a player wished to change the device they are
playing on during an on going game. There could be a case where both players
agree that a player should be able to revert back by one move they made, in
which case the game will undo the user's last move. The general use case is of
course two users playing on their devices without any flaws from start to end.

\subsection{Functional Requirements}
\begin{description}
\item [\refstepcounter{funreq} \frthefunreq:] The game will display a game board on the user's device
\item [\refstepcounter{funreq} \frthefunreq:] The user's move will be registered, recorded and reflected on the game board\\
Fit Criteria: Game board should have the user's character at the cell that they clicked on if it is a valid move
\item [\refstepcounter{funreq} \frthefunreq:] The game will determine when a player has taken control of a square\\
Fit Criteria: The array for the inner board will have one of 8 possible sets of three in a row. Three columns, three rows 
and the two diagonals
\item [\refstepcounter{funreq} \frthefunreq:] The game will determine when the square has ended in a draw\\
Fit Criteria: The array for the inner board will not have any of the 8 possible sets of three in a row. Three columns, three rows
 and the two diagonals
\item [\refstepcounter{funreq} \frthefunreq:] The game will determine when the full game is won or drawn\\
Fit Criteria: The array for the inner board will have one of 8 possible sets of three in a row on the full board. 
Three columns, three rows and the two diagonals
\item [\refstepcounter{funreq} \frthefunreq:] The game will determine in which square the next player will make their move 
based on the previous player's move\\
Fit Criteria: Clicking on any cell on an inner board that is not allowed will not result in any move
\end{description}

\section{Non-functional Requirements}

\subsection{Look and Feel Requirements}
\begin{description}
\item [\refstepcounter{lfreq} \lfthelfreq:] Game should be visually appealing\\
Fit Criteria: 90\% of surveyed users should find the colours used appealing
\item [\refstepcounter{lfreq} \lfthelfreq:] The game should have a clean graphical interface\\
Fit Criteria: 90\% of surveyed users should report that there no unnecessary graphics
\end{description}

\subsection{Usability and Humanity Requirements}
\begin{description}
\item [\refstepcounter{uhreq} \uhtheuhreq:] The game must be easy to learn\\
Fit Criteria: The average time for a user to fully learn the rules must be no more than 2 minutes after reading the rules
\item [\refstepcounter{uhreq} \uhtheuhreq:] The game should allow users to interact with the game in multiple ways\\
Fit Criteria: The click interface must always work and on computers that have touch, that should also always work
\item [\refstepcounter{uhreq} \uhtheuhreq:] There should a be a button for instructions on the rules of the game\\Fit Criteria: 90\% of users should state that they used or at least found the rules button
\end{description}

\subsection{Performance Requirements}
\begin{description}
\item [\refstepcounter{perreq} \pertheperreq:] There should be no delay in reflecting a player's move on the board\\ Fit Criteria: The maximum time for a move to be reflected on the board is 750 milliseconds
\item [\refstepcounter{perreq} \pertheperreq:] The game should process if a player has won the game or completed an inner board quickly\\
Fit Criteria: The maximum time for the board to reflect if a board is complete is 1 second
\item [\refstepcounter{perreq} \pertheperreq:] The CPU/GPU should not be overtaxed by the game\\
Fit Criteria: The maximum CPU/GPU usage on any device should be 30\%
\end{description}

\subsection{Operational and Environmental Requirements}
\begin{description}
\item [\refstepcounter{oereq} \oetheoereq:] The browser should not be using too much of the CPU power\\
Fit Criteria: The maximum browser usage on the device should be 45\%
\item [\refstepcounter{oereq} \oetheoereq:] The browser used needs HTML5\\
Fit Criteria: Browsers will need to pass the HTML5 test as provided by ~\citep{htmlTest}
\end{description}

\subsection{Maintainability and Support Requirements}
\begin{description}
\item [\refstepcounter{msreq} \msthemsreq:] Should the game fail to run, it should be repairable\\
Fit Criteria: The time to repair it should be no longer than 2 hours

\end{description}

\subsection{Security Requirements}
\begin{description}
\item [\refstepcounter{secreq} \sethesecreq:] The game should not be collecting any user information or analytics\\ Fit Criteria: There shall be 0 cookies collected with this game
\end{description}

\subsection{Cultural Requirements}
\begin{description}
\item [\refstepcounter{culreq} \cultheculreq:] The product shall not be offensive to religious or ethnic groups. \\
Fit Criteria: A survey should indicate that at least 90\% of people did not find the game offensive
\item [\refstepcounter{culreq} \cultheculreq:] The game user interface should be language ambiguous.\\
Fit Criteria: All the members of a group of people speaking different languages must be able to play the game
\end{description}

\subsection{Legal Requirements}
\begin{description}
\item [\refstepcounter{legreq} \legthelegreq:] The game will have a rating suitable for all ages\\
Fit Criteria: The ESRB rating should be E for Everyone.
\end{description}

\subsection{Health and Safety Requirements}
\begin{description}
\item [\refstepcounter{hsreq} \hsthehsreq:] The game should not result in people having epileptic episodes\\
Fit Criteria: A survey of people who play the game should indicate that none of the people had an epileptic episode as a result of playing the game
\end{description}

\section{Project Issues}

\subsection{Open Issues}
One issue we are concerned about is the adaptability of the user interface based
on the device. Some issues could arise from touch or mouse
input. Screen size of device could cause scaling problems such as having
part of the UI cut off. Lastly the user interface might not be the same based on
where the game is accessed. This could be caused due to the fact that different 
browsers render elements differently.

\subsection{Off-the-Shelf Solutions}
We will be utilizing four off the shelf solutions to assist this project 

\begin{enumerate}
	\item Moore server for web hosting
  	\item Gitlab for version control and issue tracking 
	\item Karma unit testing tool for verification 
	\item JsDoc for documentation generation
\end{enumerate}

\subsection{New Problems}
Some new problems that have arisen during the development process are user
problems such as miss click and poor internet connection. Additionally the
planned server is not powerful enough to cope with our projected growth pattern.

\subsection{Tasks}
The development cycle will follow the modified waterfall life cycle as detailed
by Dr. Smith. Refer to Figure \ref{fig:DevelopmentCycle}~\citep{Slides}. In
verification and validation, a set of test cases to ensure that all the game
features are working. Developing a thorough set of use cases as well as user
testing, the game will be tested. The design of this game will follow the
standard MVC game structure. The model will be located on the server, the view
and controller will be on each individual device. The code has been broken down
into various milestones. Please refer to the
\href{run:../../ProjectSchedule/Gantt Chart.gan}{Gantt Chart} for further
details. The final report will contain a detailed analysis of the result of the
test cases and the reviews from the test users.

\begin{figure}
  \includegraphics[width=\linewidth]{OverviewOfProcess.pdf}
  \caption{Development Cycle}
  \label{fig:DevelopmentCycle}
\end{figure}

\subsection{Migration to the New Product}
The migration to this project will involve transferring the game logic to the
server to allow for online game play if possible. The new product also will also
change the graphics and the user interface. From a MVC standpoint, the view is
the main change with the model shifting to the server. The controller will
remain the same except making it friendly for mobile devices and all touch
enabled devices.

\subsection{Risks}
There is a risk that the server that the game is hosted on could unexpectedly
crash. There is also a security risk with a device being connected insecurely to
the internet. We will ensure that only essential information is received and
there are minimal permissions

\subsection{Costs}
We plan to complete this project with a zero dollar budget. All resources used
to complete this project can be used free of charge. Additionally we plan to
host this game on the McMaster Moore server which is also available to us for
free.

\subsection{User Documentation and Training}
User documentation will be created in the form of game instructions.
Comprehensive game rules have already already been created by
mathwithbaddrawings.com~\citep{Rules} . Our game instructions will be a version
of theses rules with our own diagrams and explanations. There is no need for
user training.


\subsection{Waiting Room}
After redeveloping the Ultimate Tic Tac Toe game, we plan to add Player vs
Computer�� game mode. This AI should be able to intelligently perform moves based
on the human player's previous move. Secondly User accounts which will be able
to track player statistics such win ratio. Additionally we plan to add an online
lobby to face strangers. Players will be matched based on their player
statistics.

\subsection{Ideas for Solutions}
As discussed before we will be utilizing many off-the shelf solutions to help in
the development process. We plan to Use CSS to style the game board. To make the
game language ambiguous we will be including many images into the game
instructions. To make this game easy to maintain we shall be using
Model-��view-�controller (MVC) software architectural pattern to make it
modular.

\bibliographystyle{plainnat}

\bibliography{SRS}

\newpage

\section{Appendix}

\subsection{Symbolic Parameters}
MAX\_PLAYERS: Maximum number of concurrent players.


%TRASH
%\item [\refstepcounter{funreq} \frthefunreq:] The game to be played over the internet, the game would need to 
%communicate to both devices through means of a server.\\

\end{document}