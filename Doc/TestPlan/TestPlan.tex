\documentclass[12pt, titlepage]{article}

\usepackage{booktabs}
\usepackage{tabularx}
\usepackage{hyperref}
\usepackage{fancyhdr}
\pagestyle{fancy}

\hypersetup{
    colorlinks,
    citecolor=black,
    filecolor=blue,
    linkcolor=red,
    urlcolor=blue
}
\usepackage[round]{natbib}

\title{SE 3XA3: Test Plan\\Ultimate Tic Tac Toe}

\author{Team 3
		\\ Kunal Shah - shahk24
		\\ Pareek Ravi - ravip2
}

\date{\today}

%% Comments

\usepackage{color}

\newif\ifcomments\commentstrue

\ifcomments
\newcommand{\authornote}[3]{\textcolor{#1}{[#3 ---#2]}}
\newcommand{\todo}[1]{\textcolor{red}{[TODO: #1]}}
\else
\newcommand{\authornote}[3]{}
\newcommand{\todo}[1]{}
\fi

\newcommand{\wss}[1]{\authornote{blue}{SS}{#1}}
\newcommand{\ds}[1]{\authornote{red}{DS}{#1}}
\newcommand{\mj}[1]{\authornote{red}{MSN}{#1}}
\newcommand{\mh}[1]{\authornote{red}{MH}{#1}}
\newcommand{\cm}[1]{\authornote{red}{CM}{#1}}


% team members should be added for each team, like the following
% all comments left by the TAs or the instructor should be addressed
% by a corresponding comment from the Team

\newcommand{\tm}[1]{\authornote{magenta}{Team}{#1}}


\begin{document}

\lhead{Team 3 - Test Plan}
\rhead{Ultimate Tic Tac Toe}
\maketitle

\pagenumbering{roman}
\tableofcontents
\listoftables
\listoffigures

\begin{table}[bp]
\caption{\bf Revision History}
\begin{tabularx}{\textwidth}{p{3cm}p{2cm}X}
\toprule {\bf Date} & {\bf Version} & {\bf Notes}\\
\midrule
October 19 & 0.0 & Initial setup\\
Date 2 & 1.0 & Notes\\
\bottomrule
\end{tabularx}
\end{table}

\newpage

\pagenumbering{arabic}

\section*{Abstract} 
This document describes the Testing Plan for the Ultimate Tic Tac Toe Project.

\section{General Information}

\subsection{Purpose}
The purpose of testing this project's is to confirm all requirements that where
outlined in the Requirements Specifications document have been met and to build
confidence that the software for the project was implemented correctly.

\subsection{Scope}
The Test Plan presents a basis for testing the functionality of the re-
implementation of Ultimate Tic Tac Toe. It has the objective of proving that
Ultimate Tic Tac Toe has met the requirements specified in the Requirements
Document and to attach metrics to those requirements so that adherence to
requirements is quantifiable and can be measured. The testing plan acts as a
means to arrange testing activities. This document will present what is to be
tested of the software. It will also act as an outline of testing methods and
outline the tools that will be utilized.

\subsection{Acronyms, Abbreviations, and Symbols}
	
\begin{table}[hbp]
\caption{\textbf{Table of Abbreviations}} \label{Table}

\begin{tabularx}{\textwidth}{p{3cm}X}
\toprule
\textbf{Abbreviation} & \textbf{Definition} \\
\midrule
JS & JavaScript\\
UTTT & Ultimate Tic Tac Toe\\
\bottomrule
\end{tabularx}

\end{table}

\begin{table}[!htbp]
\caption{\textbf{Table of Definitions}} \label{Table}

\begin{tabularx}{\textwidth}{p{3cm}X}
\toprule
\textbf{Term} & \textbf{Definition}\\
\midrule
Main Board & Full game board, all clickable regions \\
Inner Board [ID] & One of 9 Tic Tac Toe boards within the Main Board\\
Cell [ID] & One of 9 Tic Tac Toe elements of an Inner Board\\
Active Board & Inner board that next player is able to play on\\
Complete & Result has been determined for an Inner board \\
In-complete & Result has yet to be determined for an Inner board \\
\bottomrule
\end{tabularx}

\end{table}	

\subsection{Overview of Document}

The Ultimate  project will re-implement the project Ultimate Tic Tac Toe. The
software will allow user to play the the game based on the rules defined by
mathwithbaddrawings.com~\citep{Rules}. All the software's requirements can be
referenced in the \href{run:../SRS/SRS.pdf}{Requirements Document}. This document
demonstrates how the game Ultimate Tic Tac Toe will be tested, the testing
schedule and the tools used.

\section{Plan}
	
\subsection{Software Description}
The software is a JavaScript implementation of the game Ultimate Tic Tac Toe. 

\subsection{Test Team}
The individuals responsible for testing are Kunal Shah and Pareek Ravi. For a
detailed breakdown of responsibilities refer to the
\href{run:../../ProjectSchedule/Gantt Chart.gan}{Gantt Chart}

\subsection{Automated Testing Approach}
The primary testing approach will be to use Karma Unit Testing to automate unit
tests. Karma will automatically play the game in a headless browser following
pre-defined moves. The unit testing framework will compare the results to
expected values

\subsection{Testing Tools}
The tool that will be utilized for this project is Karma unit testing with the
Jasmine framework. It will be used to automate the unit testing. There will be a
package.json file in the src directory to download all dependencies required for
Karma Unit Testing.\\
There will also be a Google Form used to survey user's experience and provide
and avenue for feedback and suggestions.

\subsection{Testing Schedule}	
Refer to the \href{run:../../ProjectSchedule/Gantt Chart.gan}{Gantt Chart} for
details about Testing Schedule

\section{System Test Description}
	
\subsection{Tests for Functional Requirements}

\subsubsection{User Input}

\begin{enumerate}

\item{in-test-id1\\}
\textbf {Testing if user's input being received}

Type: Functional, Dynamic, Automatic
					
Initial State: It is the start of a new game and the board is empty
					
Input: A click on any cell of any inner tic tac toe board
					
Output: That cell should have the user's character on it
					
How test will be performed: This test will be performed automatically with the use of the Karma Unit Test. In the test, an element will be clicked and then the inner html of the text will be checked to see if it is updated to the character that represents that person's turn. There will also be a check on the variable that stores the all game information to see that it is updated.
					
\item{in-test-id2\\}
\textbf{Testing if user click on invalid inner board}

Type: Functional, Dynamic, Automatic
					
Initial State: Opponent has played on a inner board and it is the user's turn

Input: Click on a tile that they are not designated to click based on the rules of the game
					
Output: The board should not change at all. The game data should not change and it is still the user's turn
					
How test will be performed: This test will be done automatically with theuse of the Karma Unit Test. When the use clicks on a inner board that they cannot select, the javascript will prevent anything from changing in the game data

\item{in-test-id3\\}
\textbf{Testing if user's input on cell already clicked}

Type: Functional, Dynamic, Automatic
					
Initial State: Opponent has played on a inner board and it is the user's turn
					
Input: Click on a tile that has already been clicked previously
					
Output: The board should not change at all. The game data should not change and it is still the user's turn
					
How test will be performed: This test will be performed automatically with the use of the Karma Unit Test. In the test, an element in the inner board they are meant to click is already selected previously will be clicked. There will be no change to the board as the cell was previously selected. The script will not change anything as that cell is not a null value, it has the character representing the player that selected it.

\end{enumerate}

\subsubsection{Game Logic}

\begin{enumerate}

\item{log-test-id1\\}
\textbf{Test if inner board gets completed}

Type: Functional, Dynamic, Automatic
					
Initial State: Inner board is almost completed by user. It is their turn to play in the inner board where they will complete it. 
					
Input: Clicks on the cell that will complete the inner board
					
Output: The inner board will be marked with the character representing the user
					
How test will be performed: This test will be performed automatically with the use of the Karma Unit Test. The cell element will be clicked by the tester. The logic will check if there is any three in a row in that inner board that the player just played in. If there is a three in a row, the entire board is deemed as completed and marked as so. The check for a three in a row is to check all 8 possible win scenarios, i.e. three rows, three columns and two diagonals

\item{log-test-id2\\}
\textbf{Test if inner board ends in draw}

Type: Functional, Dynamic, Automatic
					
Initial State: The board is in a state where an inner board only has 1 cell available to click and it will result in that board being a draw

Input: Click on the only available cell
					
Output: The inner board will be marked with the character '-' meaning it is a draw
					
How test will be performed: This test will be performed automatically with the use of the Karma Unit Test. The cell element will be clicked by the tester. The logic will check all 8 possible cases for a completed three in a row. If non exist and the inner board is full it is classified as a draw.

\item{log-test-id3\\}
\textbf{Test if next move can be made on any incomplete inner board}

Type: Functional, Dynamic, Automatic
					
Initial State: Game board is partially filled with one inner board completed

Input: Click at a cell coresponding to a completed inner board
					
Output: All incomplete inner boards active
					
How test will be performed: When the click is made on the inner board, the background of all inner boards that are not completed is set to blue. Based on the array which contains a map of the board, a loop through all the inner board elements and check their background colors. Any inner board that is not complete will have a background style blue.

\end{enumerate}

\subsubsection{Game Logistics}

\begin{enumerate}

\item{logistic-test-id1\\}
\textbf{Test if game launches}

Type: Functional, Dynamic, Manual
					
Initial State: User is in the file explorer
					
Input: User launches the html file in browser
					
Output: The game launches in all browsers and shows an empty UTTT game board. 
					
How test will be performed: The user will launch the game from their file explorer. If they are able to see a blank UTTT board, the game has launched.

\item{logistic-test-id2\\}
\textbf{Test if user input shows in window}

Type: Functional, Dynamic, Manual
					
Initial State: User has just openned game
					
Input: User clicks on any cell
					
Output: Cell they clicked on should change appearance
					
How test will be performed: If the user's click was registered, it would indicate that to the user by changing the cell they clicked on to the character that represents them. This will be seen graphically.

\end{enumerate}

\subsection{Tests for Nonfunctional Requirements}

\subsubsection{Look and Feel}

\begin{enumerate}

\item{laf-test-id1\\}
This will be tested by survey question~\ref{question:q4}.

Pass: 4.5/5 average rating on this question
\item{laf-test-id2\\}
This will be tested by survey question~\ref{question:q10}.

Pass 4.5/5 average rating for this question. The details will provide more information on repairs needed
\item{laf-test-id3\\}
This will be tested by survey question~\ref{question:q11}.

Pass 4.5/5 average rating for this question. The details will provide more information on repairs needed
\end{enumerate}

\subsubsection{Ease of Use}

\begin{enumerate}

\item{eou-test-id1\\}
This will be tested by survey question~\ref{question:q1}.

Pass: 4.5/5 average rating on this question
\item{eou-test-id2\\}
This will be tested by survey question~\ref{question:q2}.

Pass 4.5/5 average rating for this question.
\item{eou-test-id3\\}
This will be tested by survey question~\ref{question:q3}.

Pass 4.5/5 average rating for this question.
\end{enumerate}

\subsubsection{Environmental Requirements}

\begin{enumerate}

\item{er-test-id1\\}
This will be tested by survey question~\ref{question:q6}.

Pass: 4.5/5 average rating on this question. The details could lead to optimization
\end{enumerate}

\subsubsection{Performace Requirements}

\begin{enumerate}

\item{pr-test-id1\\}
This will be tested by survey question~\ref{question:q5}.

Pass: 4.5/5 average rating on this question.
\item{pr-test-id2\\}
This will be tested by survey question~\ref{question:q7}.

Pass: 4.5/5 average rating on this question. The details could lead to optimization
\end{enumerate}

\subsubsection{Safety Requirements}

\begin{enumerate}

\item{safreq-test-id1\\}
This will be tested by survey question~\ref{question:q9}.

Pass: 4.5/5 average rating on this question.
\end{enumerate}

\subsubsection{Cultural Requirements}

\begin{enumerate}

\item{culreq-test-id1\\}
This will be tested by survey question~\ref{question:q8}.

Pass: 4.5/5 average rating on this question. The details could lead to optimization
\end{enumerate}

\subsubsection{Legal Requirements}

\begin{enumerate}

\item{legreq-test-id1\\}
Description: This game should have a rating suitable for all ages to play according to ESRB
How: ESRB will rate the game based on their standards which are accepted universally
Pass: The game has a rating of E for everyone.
\end{enumerate}

\subsubsection{Security Requirements}

\begin{enumerate}

\item{secreq-test-id1\\}
Description: When the game is implemented on a server, no personal data should be transfered
How: Check all the packages that are transfered over the server
Pass: Only game data is passed, no personal data
\end{enumerate}

\section{Tests for Proof of Concept}

\subsection{User Input}
		
\paragraph{User inputs from the same local machine}

\begin{enumerate}

\item{Test first click\\}

Type: Manual
					
Initial State: On Load
					
Input: click on Inner Board [B00] Cell [1] 
					
Output: Player1 symbol ( either X or O )
					
How test will be performed: User clicks input on game page loaded on an browser. User watches for graphical response.

\item{Set Active Board\\}

Type: Manual
					
Initial State: On Load
					
Input: User click Inner Board [B03] Cell [5] 
					
Output:  Active Board set to Inner Board [B11]
					
How test will be performed: User clicks input on game page loaded on an browser. User watches for graphical response. 

\item{All incomplete Inner Boards active when player sets complete inner board active \\}

Type: Manual
					
Initial State: One move till Player 1 completes inner board [B02]
					
Input: User click Inner Board [B02] Cell [3] 
					
Output:  all Inner Boards excluding Inner Board [B02] show blue background colour
					
How test will be performed: Users clicks the following sequence on game page loaded on an browser. User watches for graphical response.
\begin{enumerate}
	\item Player 1 clicks on Inner Board [B02] Cell [5]
	\item Player 2 clicks on Inner Board [B11] Cell [3]
	\item Player 1 clicks on Inner Board [B02] Cell [7]
	\item Player 2 clicks on Inner Board [B20] Cell [3]
	\item Player 1 clicks on Inner Board [B02] Cell [3]
\end{enumerate}

\subsection{Game Logic}

\item{Complete Inner Board\\}

Type: Manual
					
Initial State: One move till Player 1 completes inner board [B01]
					
Input: User click Inner Board [B01] Cell [3] 
					
Output:  Inner Board [B01] displays Player 1 symbol
					
How test will be performed: Users clicks the following sequence on game page loaded on an browser. User watches for graphical response.
\begin{enumerate}
	\item Player 1 clicks on Inner Board [B02] Cell [5]
	\item Player 2 clicks on Inner Board [B11] Cell [3]
	\item Player 1 clicks on Inner Board [B02] Cell [7]
	\item Player 2 clicks on Inner Board [B20] Cell [3]
	\item Player 1 clicks on Inner Board [B02] Cell [3]
\end{enumerate}

\item{Complete Inner Board with draw(or tie) \\}

Type: Manual
					
Initial State: One move till Player completes inner board
					
Input: User click to cause Inner Board to draw 
					
Output:  Inner Board displays dash to indicate draw
					
How test will be performed: User clicks input on game page loaded on an browser. User watches for graphical response.

\item{Win Full Game\\}

Type: Manual
					
Initial State: One move till Player 1 wins full game
					
Input: User click cell to complete last inner board needed to win Main Board
					
Output:  Browser alert with player that won ( X or O )
					
How test will be performed: Users clicks the following sequence on game page loaded on an browser. User watches for graphical response.

\end{enumerate}
	
\section{Comparison to Existing Implementation}
	
				
\section{Unit Testing Plan}
		
\subsection{Unit testing of internal functions}
		
\subsection{Unit testing of output files}		

\bibliographystyle{plainnat}

\bibliography{TestPlan}

\newpage

\section{Appendix}
Karma JS Installation Tutorial and Example source code~\citep{Karma}

\subsection{Symbolic Parameters}

The definition of the test cases will call for SYMBOLIC\_CONSTANTS.
Their values are defined in this section for easy maintenance.

\subsection{Usability Survey Questions?}
Answers will be rated from 1 to 5, 5 being the highest
\begin{enumerate}
\item
Is it easy to find the rules of the game? \label{question:q1}
\item
Are the rules easy to understand? \label{question:q2}
\item
Is it clear which player's turn it is? (X or O) \label{question:q3}
\item
Is the color pallet visually appealing? \label{question:q4}
\item
Is the response time satisfying? \label{question:q5}
\item
Was the game able to run on your browser? Please indicate the browser you used \label{question:q6}
\item
Was the game able to run smoothly on your device. Please indicate the computer used and the manufactured year \label{question:q7}
\item
Were you offended by anything in the game. Please provide details \label{question:q8}
\item
Did you encounter epileptic symptoms while playing the game \label{question:q9}
\item
Did anything unexpected appear on the screen while playing. Please describe \label{question:q10}
\item
Is the entire game board visible on the screen? Are there any aspects that are cut off? Please describe \label{question:q11}


\end{enumerate}

\end{document}